\documentclass[12pt, a4paper, titlepage]{article}

\usepackage[portuguese]{babel}
\usepackage{amsmath}
\usepackage{tabularx}
\usepackage{chngpage}

\title{\textbf{Orchestrador de Tarefas}}
\author{
    \begin{tabular}{ll}
        Humberto Gil Azevedo Sampaio Gomes & A104348 \\
        José António Fernandes Alves Lopes & A104541 \\
        José Rodrigo Ferreira Matos        & A100612 \\
    \end{tabular}
}
\date{\today}

\begin{document}
\maketitle
\pagebreak

\section{Protocolo}
    O protocolo implementado no nosso projeto define um conjunto de instruções usadas para a
    correta identificação e estruturação das mensagens enviadas entre cliente e servidor. De seguida
    apresentamos uma definição para cada tipo de mensagem definido no nosso protocolo, seguido de
    uma figura ilustrativa que apresenta graficamente a estrutura de cada tipo de mensagem. A ordem
    de apresentação dos diferentes campos em cada parágrafo é condizente com a ordem de apresentação
    gráfica.

    Do lado do cliente podemos esperar o envio de 4 tipos diferentes de mensagem, representados no
    protocolo a partir das seguintes etiquetas bem definidas:
    \texttt{PROTOCOL\_C2S\_SEND\_PROGRAM}, \texttt{PROTOCOL\_C2S\_SEND\_TASK},
    \texttt{PROTOCOL\_C2S\_TASK\_DONE} e \texttt{PROTOCOL\_C2S\_STATUS}.

    Mensagens do tipo \texttt{PROTOCOL\_C2S\_SEND\_PROGRAM} têm como função o transporte de um
    programa único a ser executado pelo servidor, enquanto que mensagens
    \texttt{PROTOCOL\_C2S\_SEND\_TASK} devem transportar múltiplos programas a ser executados em
    \emph{pipeline}. Ambos os tipos possuem uma parametrização idêntica, sendo compostos por 5
    campos; o identificador do tipo de mensagem, que terá de ser uma das duas opções apresentadas
    anteriormente; o identificador do processo que enviou a mensagem (\emph{process identifier});
    uma marca temporal do momento em que a mensagem foi enviada; o tempo de execução previsto pelo
    cliente em milissegundos, e, por fim, o comando introduzido pelo utilizador ao executar o
    cliente.

    \begin{center}
        \abovedisplayskip=-1pt
        \texttt{PROTOCOL\_C2S\_SEND\_(PROGRAM / TASK)}
        $$\overleftrightarrow{
            \begin{tabular}{|c|c|c|c|c|}
                \hline
                    type & client\_pid & time\_sent & expected\_time & command\_line
                    \rule[-2ex]{0pt}{6ex}\\
                \hline
            \end{tabular}
        }$$
    \end{center}

    Já mensagens do tipo \texttt{PROTOCOL\_C2S\_TASK\_DONE} são responsáveis por avisar o servidor
    do término da execução de uma tarefa realizada por um processo filho, que neste contexto é
    tratado como cliente. A estrutura de mensagens deste tipo é caracterizada por 5 campos; o
    identificador do tipo de mensagem, obrigatoriamente \texttt{PROTOCOL\_C2S\_TASK\_DONE}; a
    posição em que a tarefa foi agendada; uma marca temporal de quando a tarefa terminou a sua
    execução; e, por fim, duas etiquetas, a primeira verifica se a tarefa executada é, ou não, um
    pedido de estado do servidor, e a segunda verifica se a execução da tarefa resultou, ou não, num
    erro.

    \begin{center}
        \abovedisplayskip=-1pt
        \texttt{PROTOCOL\_C2S\_SEND\_TASK\_DONE}
        $$\overleftrightarrow{
            \begin{tabular}{|c|c|c|c|c|}
                \hline
                    type & slot & time\_ended & is\_status & error
                    \rule[-2ex]{0pt}{6ex}\\
                \hline
            \end{tabular}
        }$$
    \end{center}

    Por último, mensagens do tipo \texttt{PROTOCOL\_C2S\_STATUS} são responsáveis por transportar
    pedidos de estado do servidor. A sua estrutura é bastante simples, sendo apenas compostas por
    2 campos; o identificador do tipo de mensagem, obrigatoriamente \texttt{PROTOCOL\_C2S\_STATUS},
    e o identificador de processo do cliente.

    \begin{center}
        \abovedisplayskip=-1pt
        \texttt{PROTOCOL\_C2S\_STATUS}
        $$\overleftrightarrow{
            \begin{tabularx}{0.4\textwidth}
                {|>{\centering\arraybackslash}X|>{\centering\arraybackslash}X|}
                \hline
                    type & client\_pid
                    \rule[-2ex]{0pt}{6ex}\\
                \hline
            \end{tabularx}
        }$$
    \end{center}

    Do lado do servidor podemos esperar 3 tipos diferentes de mensagens:
    \texttt{PROTOCOL\_S2C\_ERROR}, \texttt{PROTOCOL\_S2C\_TASK\_ID} e \texttt{PROTOCOL\_S2C\_STATUS}.
    O primeiro tipo de mensagem, \texttt{PROTOCOL\_S2C\_ERROR}, é responsável por avisar o cliente
    de erros ocorridos durante o processamento de uma tarefa. Para tal, apenas é necessário
    transportar na mensagem o identificador do tipo de mensagem, neste caso
    \texttt{PROTOCOL\_S2C\_ERROR}, e o erro em si.

    \begin{center}
        \abovedisplayskip=-1pt
        \texttt{PROTOCOL\_S2C\_ERROR}
        $$\overleftrightarrow{
            \begin{tabularx}{0.4\textwidth}
                {|>{\centering\arraybackslash}X|>{\centering\arraybackslash}X|}
                \hline
                    type & error
                    \rule[-2ex]{0pt}{6ex}\\
                \hline
            \end{tabularx}
        }$$
    \end{center}

    Mensagens do tipo \texttt{PROTOCOL\_S2C\_TASK\_ID} estão encarregues de informar clientes acerca
    do identificador da tarefa que requisitaram ao servidor. Estas são, mais uma vez, de simples
    estruturação, contendo apenas o identificador do tipo de mensagem, obrigatoriamente
    \texttt{PROTOCOL\_S2C\_TASK\_ID}, e o identificador do processo agendado pelo servidor.

    \begin{center}
        \abovedisplayskip=-1pt
        \texttt{PROTOCOL\_S2C\_TASK\_ID}
        $$\overleftrightarrow{
            \begin{tabularx}{0.4\textwidth}
                {|>{\centering\arraybackslash}X|>{\centering\arraybackslash}X|}
                \hline
                    type & id
                    \rule[-2ex]{0pt}{6ex}\\
                \hline
            \end{tabularx}
        }$$
    \end{center}

    Por fim, resta detalhar as mensagens \texttt{PROTOCOL\_S2C\_STATUS}, cuja função é informar
    clientes acerca do estado da respetiva tarefa que requisitaram do servidor. A estruturação deste
    tipo de mensagens é mais complexa, sendo cada mensagem composta por 9 campos; o tipo de
    mensagem, obrigatoriamente \texttt{PROTOCOL\_S2C\_STATUS}; o estado da tarefa; o identificador
    da tarefa a que a mensagem se refere; uma etiqueta que identifica se um erro foi detetado
    durante a execução da tarefa; o tempo, em microssegundos, que o processo demorou a chegar ao
    servidor; o tempo que a tarefa esteve em fila de espera, que apenas se aplica a tarefas já
    terminadas, ou em execução; o tempo de execução da tarefa, aplicado apenas a tarefas que já
    terminaram a sua execução; o tempo que o processo filho que executou a tarefa demorou a dizer ao
    servidor que a tarefa tinha sido terminada, e, finalmente, a linha de comando da tarefa
    submetida. \\

    \begin{adjustwidth}{-1in}{-1in}
    \begin{center}
        \abovedisplayskip=-1pt
        \texttt{PROTOCOL\_S2C\_STATUS}
        $$\overleftrightarrow{
            \begin{tabular}{|c|c|c|c|c|c|c|c|c|}
                \hline
                    type &
                    status &
                    id &
                    error &
                    time\_c2s\_fifo &
                    time\_waiting &
                    time\_executing &
                    time\_s2s\_fifo &
                    command\_line
                    \rule[-2ex]{0pt}{6ex}\\
                \hline
            \end{tabular}
        }$$
    \end{center}
    \end{adjustwidth}

    O estado de uma tarefa pode ser descrito de 3 formas, a tarefa pode estar em execução, pode já
    ter terminado, ou pode ainda estar em fila de espera. Representamos estes 3 estados,
    respetivamente, a partir das seguintes etiquetas bem definidas:
    \texttt{PROTOCOL\_TASK\_STATUS\_EXECUTING}, \texttt{PROTOCOL\_TASK\_STATUS\_DONE} e
    \texttt{PROTOCOL\_TASK\_STATUS\_QUEUED}.

\end{document}
